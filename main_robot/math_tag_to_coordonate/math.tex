\documentclass{article}
\usepackage{amsmath}
\usepackage{amsfonts}
\usepackage{graphicx}
\graphicspath{ {./} }

\title{Maths pour localiser un ArUco}
\author{Ayman AIT HADDOU }

\begin{document}
\maketitle
\section{Explications}
On considère un tag ArUco carré $(ABDC)$ comme dans l'illustration ci dessous. \\
On a la caméra en $P$ qui capture une image du tag.\\
À l'aide de opencv, on retrouve les vecteurs $\vec{A_c}$, $\vec{B_c}$, $\vec{C_c}$ et $\vec{D_c}$.\\
Où si l'on pose $\mathcal{P}$ le plan orthogonal à la direction de la caméra du coté de la direction de la caméra et à distance 1 de $P$, $\vec{X_c}$ est le vecteur de $P$ au point de $(XP) \cap \mathcal{P}$.\\
On cherche dans un premier temps $\vec u$ un vecteur directeur de $(A,P,B) \cap (C,P,D)$ qui est collinéaire à $\vec AB$ et à $\vec AC$.\\
On va ensuite chercher le vecteur $\vec{u}_{AB}$ (resp. $\vec{u}_{CD}$) colinéaire à $\vec{AB}$ (resp. $\vec{CD}$) entre $P + \vec{A}_c$ à un point de $(PB)$ (resp. entre $P + \vec{C}_c$ à un point de $(PD)$)\\
Ensuite comme on connait la vraie longueur des côtés du carré $(ABDC)$ en comparant $||\vec{u}_{AB}||$ (resp. $||\vec{u}_{CD}||$) on peut rerouver le facteur par lequel multiplier pour trouver $\vec{PA}$ (resp. $\vec{PC}$) à partir de $\vec{A}_c$ (resp. $\vec{C}_c$) et $\vec{AB}$ (resp. $\vec{CD}$) à partir de $\vec{u}_{AB}$ (resp. $\vec{u}_{CD}$)\\
\includegraphics[scale=0.25]{illustration.png}


\section{Recherche de $\vec{u}$}
Soit $\vec u$ un vecteur directeur de $(A,P,B) \cap (C,P,D)$.
On pose $\vec{n_1} = \vec{A_c} \wedge \vec{B_c}$ normal à $(A,P,B)$ et $\vec{n_2} = \vec{C_c} \wedge \vec{D_c}$ normal à $(C,P,D)$.\\
On prend alors $\vec u = \vec{n}_1 \wedge \vec{n}_2$

\section{Calcul de $\vec{u}_{AB}$}
On raisonne uniquement sur le points et les vecteurs du plan $(APB)$ on raisonnera de même pour $C$ et $D$.\\
\\Dans le plan vectoriel parrallèle à $(APB)$ (sa direction), on cherche $\vec{u}_{AB}$ le projeté de $\vec{B_c}-\vec{A_c}$ sur $Vect(\vec u)$ parallèlement à $Vect(\vec{B_c})$. (regarder l'illustration pour comprendre le sens géometrique)\\
\\
On pose avec $\alpha, \beta \in \mathbb{R}$.
$$
\vec{A}_c = \alpha \vec{u} + \beta \vec{B}_c
$$
On a 
$$
\left\lbrace\begin{matrix}
\alpha \vec u \cdot \vec u &+ \beta \alpha \vec{B}_c \cdot \vec u &= \vec{A}_c \cdot \vec u \\
\alpha \vec u \cdot \vec{B}_c &+ \beta \alpha \vec{B}_c \cdot \vec{B}_c &= \vec{A}_c \cdot \vec{B}_c
\end{matrix}\right.
$$
On pose alors
$$
\begin{matrix}
a = \vec u \cdot \vec u & d = \vec{A}_c \cdot \vec u \\
b = \vec u \cdot \vec{B}_c & e = \vec{A}_c \cdot \vec{B}_c \\
c = \vec{B}_c \cdot \vec{B}_c &\\
\end{matrix}
$$
on a
$$
\begin{pmatrix}
\alpha \\ \beta
\end{pmatrix} = \begin{pmatrix}
a & b \\
b & c
\end{pmatrix}^{-1} \begin{pmatrix}
e \\ f
\end{pmatrix}
$$
et enfin 
$$
\vec{u}_{AB} = \alpha \vec u
$$


\section{Comparaison de $\vec{u}_{AB}$ à $\vec{AB}$ ce qui donne $\vec{PA}$}
On remarque
$$
AB\vec{A}_c = ||\vec{u}_{AB}||\vec{PA}
$$
On a enfin
\begin{align*}
\vec{PA} = \frac{AB}{||\vec{u}_{AB}||}\vec{A}_c \\
\vec{PB} = \frac{AB}{||\vec{u}_{AB}||}(\vec{A}_c + \vec{u}_{AB})
\end{align*}
De même on peut trouver $\vec{PC}$ et $\vec{PD}$.
\end{document}